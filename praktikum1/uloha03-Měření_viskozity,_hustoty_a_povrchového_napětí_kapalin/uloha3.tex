\documentclass[a4paper,11pt]{article}

% Kodovani (cestiny) v dokumentu: utf-8
%\usepackage[cp1250]{inputenc}	% Omezena stredoevropska kodova stranka, pouze MSW.
\usepackage[utf8]{inputenc}	% Doporucujeme pouzivat UTF-8 (unicode).

\usepackage[margin=2cm]{geometry}
\newtoks\jmenopraktika \newtoks\jmeno \newtoks\datum
\newtoks\obor \newtoks\skupina \newtoks\rocnik \newtoks\semestr
\newtoks\cisloulohy \newtoks\jmenoulohy
\newtoks\tlak \newtoks\teplota \newtoks\vlhkost

\jmenopraktika={Fyzikální praktikum 1}
\jmeno={Lukáš Lejdar}
\datum={16. dubna}
\obor={F}
\skupina={Út 16:00}

\cisloulohy={3}
\jmenoulohy={Měření viskozity, hustoty a povrchového napětí kapalin}
\tlak={101{,}35}
\teplota={21,1}
\vlhkost={47,7}

%%%%%%%%%%% Uzitecne balicky:
\usepackage[czech]{babel}

\usepackage{graphicx}
\usepackage{amsmath}
\usepackage{xspace}
\usepackage{url}
\usepackage{indentfirst}
\usepackage{wrapfig}
\usepackage{xcolor}

%%%%%% Zamezeni parchantu:
\widowpenalty 10000 \clubpenalty 10000 \displaywidowpenalty 10000
%%%%%% Parametry pro moznost vsazeni vetsiho poctu obrazku na stranku
\setcounter{topnumber}{3}	  % max. pocet floatu nahore (specifikace t)
\setcounter{bottomnumber}{3}	  % max. pocet floatu dole (specifikace b)
\setcounter{totalnumber}{6}	  % max. pocet floatu na strance celkem
\renewcommand\topfraction{0.9}	  % max podil stranky pro floaty nahore
\renewcommand\bottomfraction{0.9} % max podil stranky pro floaty dole
\renewcommand\textfraction{0.1}	  % min podil stranky, ktery musi obsahovat text
\intextsep=8mm \textfloatsep=8mm  %\intextsep pro ulozeni [h] floatu a \textfloatsep pro [b] or [t]

% Tecky za cisly sekci:
\renewcommand{\thesection}{\arabic{section}.}
\renewcommand{\thesubsection}{\thesection\arabic{subsection}.}
% Jednopismenna mezera mezi cislem a nazvem kapitoly:
\makeatletter \def\@seccntformat#1{\csname the#1\endcsname\hspace{1ex}} \makeatother
%
\newcommand{\vsn}[4]{\ensuremath{#1 =} #2(#3)\,#4}
\newcommand{\vrn}[6]{\ensuremath{#1 =} (#2 $\pm$ #3)\,#4 ($p=$ #5\,\%, $\nu=$ #6)}


%%%%%%%%%%%%%%%%%%%%%%%%%%%%%%%%%%%%%%%%%%%%%%%%%%%%%%%%%%%%%%%%%%%%%%%%%%%%%%%
% Zacatek dokumentu
%%%%%%%%%%%%%%%%%%%%%%%%%%%%%%%%%%%%%%%%%%%%%%%%%%%%%%%%%%%%%%%%%%%%%%%%%%%%%%%

\begin{document}

\thispagestyle{empty}

{
\begin{center}
\sf 
{\Large Ústav fyziky a technologií plazmatu Přírodovědecké fakulty Masarykovy univerzity} \\
\bigskip
{\huge \bfseries FYZIKÁLNÍ PRAKTIKUM} \\
\bigskip
{\Large \the\jmenopraktika}
\end{center}

\bigskip

\sf
\noindent
\setlength{\arrayrulewidth}{1pt}
\begin{tabular*}{\textwidth}{@{\extracolsep{\fill}} l l}
\large {\bfseries Zpracoval:}  \the\jmeno & \large  {\bfseries Naměřeno:} \the\datum\\[2mm]
\large  {\bfseries Obor:} \the\obor  \hspace{40mm}  {\bfseries Skupina:} \the\skupina %
&\large {\bfseries Testováno:}\\
\\
\hline
\end{tabular*}
}

\bigskip

{
\sf
\noindent \begin{tabular}{p{4cm} p{0.6\textwidth}}
\Large  Úloha č. {\bfseries \the\cisloulohy:} \par
\smallskip
$T=\the\teplota$~$^\circ$C \par
$p=\the\tlak$~kPa \par
$\varphi=\the\vlhkost$~\%
&\Large \bfseries \the\jmenoulohy  \\[2mm]
\end{tabular}
}

\vskip1cm

\section{Úvod}

Úloha je zaměřená na základní mechanické vlastnosti kapalin a metody jejich měření. Pokusím~se~ty nejdůležitější veličiny nejdřív krátce shrnout. \\

I v kapalině probíhá efekt podobný tření. Molekuly proudící blízko sebe spolu různě interagují a mají tendenci svoji rychlost srovnávat. Tento efekt popisuje Newtonův zákon pro laminárně proudící kapaliny zavedením \textbf{dynamické viskozity} $\eta$, jako konstantu přímé úměry

\begin{equation}
  \tau = \eta \frac{dv_x}{dy},
\end{equation}

kde $\tau$ je smykové napětí mezi vrstvami kapaliny a  $\frac{dv_x}{dy}$ je derivace rychlosti proudění kapaliny ve
směru normály k rovině smykového napětí $\tau$. S dynamickou viskozitou se často setkáme v podílu s hustotou kapaliny. Konvenčně tuto veličinu definujeme jako \textbf{Kinematickou viskozitu} $\nu$

\begin{equation}
\nu = \frac{\eta}{\rho}
\end{equation}

Další důležitou vlastností kapalin je tzv. \textbf{povrchové napětí} $\sigma$. Definujeme ho jako sílu působící v rovině povrchu kapaliny, kolmo na libovolnou délkovou jednotku v povrchu kapaliny, vztaženou na tuto délku.

\begin{equation}
  \sigma = \frac{F}{L}
\end{equation}

Podle Fowkese a van Osse et al. působí v povrchové vrstvě různé mezimolekulární síly, které aditivně přispívají k celkovému povrchovému napětí. Většinu z nich lze zahrnout do dvou větších kategorií acidobazických interakcí a Van der Walsových sil.

\begin{equation}
  \sigma = \sigma^{\text{lw}} + \sigma^{\text{ab}}
\end{equation}

\newpage

\section{Teorie}

\subsection{Absolutní měření viskozity kapalin}

Pokud kapalina proudí v kapiláře laminárně, platí zjednodušený tvar Hagenovy – Poiseuillovy rovnice

\begin{equation}
\frac{dV}{dt} = v \pi R^2 = \frac{\pi R^4 \Delta p }{8 \eta L} 
\end{equation}

\noindent
kde $\Delta$p je tlakový spád mezi konci trubice, L délka trubice, R poloměr trubice a $\frac{dV}{dt}$ okamžitý průtokový objem. Mariottova láhev z obrázku 1 dokáže zajistit stálý tlakový spád, čímž se vztah dál zjednoduší na tvar

\begin{equation}
\eta = \frac{\pi R^4 \Delta p t}{8 V L}.
\end{equation}

\begin{figure}[htpb]
  \centering
  \includegraphics[width=0.4\textwidth]{mariottova_lahev.jpg}
  \caption{Mariottova láhev s kapilárou pro absolutní měření viskozity.}
\end{figure}

\subsection{Ubbelohdeho viskozimetr}

Ubbelohdeho viskozimetr na obrázku 2 je právě připravený k měření. Na kapiláře se díky trubici~A vytváří tlakový rozdíl $\Delta p$ = $h(t) \rho g + p_{Atm} - p_{Atm}$ a kapalina z trubice B tak může protékat do spodních baněk. Pokud je proudění laminární, platí opět Poiseuillův vztah (4).

\begin{figure}[htpb]
  \centering
  \includegraphics[width=0.13\textwidth]{ubbelhdeh.jpg}
  \caption{Ubbelohdeho kapilární viskozimetr}
\end{figure}

Uvažujme časový interval od momentu $t_1$, kdy hladina v trubici B míjí rysku 1 do chvíle $t_2$, kdy míjí rysku 2. 

\begin{align}
  \eta \frac{dV}{dt} &= \frac{\pi R^4}{8 L} \Delta p \\
  \eta A(h) dh &= \frac{\pi R^4}{8 L} h \rho g dt \\
  \frac{\eta}{\rho} \int_{h_1}^{h_2} \frac{A(h)}{h} dh &= \frac{\pi R^4}{8 L} g \int_{t_1}^{t_2} dt 
\end{align}

Všechny konstanty vystupující na pravé straně jsou nezávislé na vlastnostech měřené kapaliny, stejně jako Integrál na straně levé. Můžu tak psát

\begin{equation}
\nu = K(t_2-t_1),
\end{equation}

kde $\nu$ je kinematická viskozita, $t_2-t_1$ čas za který odteče kapalina od rysky 1 k rysce 2 a K časová konstanta viskozimetru. Časovou konstantu je pro měření nejdřív potřeba zkalibrovat pomocí kapaliny se známou kinematickou viskozitou.

\subsection{Měření povrchového napětí du Noüyho metodou}

Du Noüyho metoda spočívá v měření síly působící podél obvodu ponořeného objektu. Na podvěsné váhy pověsím železný kroužek, váhy vytáruji a pod kroužek položím misku s kapalinou na desku s vertikálním posuvem. Misku začnu pomalu zvedat, dokud se celý kroužek neponoří a pak ji zase pomalu vytahuji. Celý proces by měl probíhat jako na obrázku 3.

\begin{figure}[htpb]
  \centering
  \includegraphics[width=0.4\textwidth]{nouyh.jpg}
  \caption{Závislost síly působící na kroužek při jeho ponoru a vytahování. Převzato
z http://www.attension.com}
\end{figure}

Povrchové napětí bude odpovídat největší síle kterou dokázala kapalina působit na délku kroužku

\begin{equation}
\sigma = \frac{F_{\text{max}}}{4 \pi R} \cdot f (\frac{R^3}{V}, \frac{R}{r}),
\end{equation}

\noindent
kde R je poloměr kroužku a $f (\frac{R^3}{V}, \frac{R}{r})$ HarkinsůvJordanův korekční faktor, který je tabelován pro bezrozměrné veličiny $\frac{R^3}{V}$ a $\frac{R}{r}$ ; $V=\frac{F_{\text{max}}}{(\rho - \rho_v)g}$ a r je poloměr drátku.

\subsection{Měření hustoty metodou ponorného tělíska}

Tato metoda využívá Archimedova zákona pro měření hustoty kapaliny. Změříme vztlak působící na plně ponořené těleso v kapalině o známé hustotě a v kapalině s neznámou hustotou. Výslednou hdonotu získáme ze vztahu

\begin{equation}
  \rho = \frac{m}{m_{\text{známé}}} \rho_{\text{známé}}
\end{equation}

\subsection{Měření hustoty pyknometrem}

Pyknometr je skleněná nádoba, kterou lze naplnit velmi přesným, ale neznámým objemem kapaliny. Zvlášť zvážím hmotnost m prázdného pyknometru, hmotnost $m_1$ pyknometru naplněného kapalinou o neznámé hustotě $\rho_1$ a hmotnost $m_2$ pyknometru naplněného kapalinou o známé hustotě $\rho_2$.

\begin{equation}
  \rho_1 = (\rho_2 - \rho_1) \frac{m_1 - m}{m_2 - m} + \rho_{\text{vzduchu}}
\end{equation}

\subsection{Měření složek povrchového napětí metodou kontaktního úhlu přisedlé kapky}

Povrchové napětí nejčastěji měříme na rozhraní kapalina - vzduch, jelikož vzduch nereaguje ani s kapalinou, ani sám se sebou. Nevznikne žádný příspěvek od molekul vzduchu a nic neoponuje přitažlivým silám mezi molekulami kapaliny, které se tím v součtu projeví všechny. Dupreho rovnice (14) naopak popisuje případ, kdy obě látky na rozhraní mají vlastní povrchové napětí $
\sigma_s$,  $\sigma_l$ a zároveň spolu interagují.

\begin{equation}
  \sigma_{\text{sl}} =
  \sigma_{\text{s}} + \sigma_{\text{l}} - 2 \sqrt{\sigma_{\text{s}}^{\text{lw}}\sigma_{\text{l}}^{\text{lw}}} - 2\sqrt{\sigm
a_{\text{s}}^{\text{ab}}\sigma_{\text{l}}^{\text{ab}}} \\
\end{equation}

Cílem měření je zjistit, jak se známá hodnota povrchového napětí $\sigma_{\text{l}}$ rozkládá do složek $\sigma_{\text{l}}^{\text{lw}}$, $\sigma_{\text{l}}^{\text{ab}}$ ze vztahu (4) a jak bude podle vztahu (14) reagovat s jinými materiály.  Kapka nanesená na vhodný povrch jako na obrázku~4, je právě co hledáme. Pro úhel $\theta$, který svírá tečna k profilu kapky v místě styku všech tří fází platí Youngova rovnice \\

\begin{equation}
\sigma_{\text{s}} = \sigma_{\text{sl}} + \sigma_{\text{l}}\cos(\theta),
\end{equation}

\noindent
kde $\sigma_{\text{l}}$ a $\sigma_{\text{s}}$ jsou povrchové mezifázové energie kapaliny a pevné látky vůči páře kapaliny a $\sigma_{\text{sl}}$ mezifázová povrchové energie rozhraní kapalina – pevná látka.

\begin{figure}[htpb]
  \centering
  \includegraphics[width=0.5\textwidth]{young.jpg}
  \caption{Youngova rovnováha na rozhraní tří fází u přisedlé kapky}
\end{figure}

Kombinací Youngovy a Dupreho rovnice dostaneme rovnici Youngovu-Dupreho:

\begin{equation}
  (1 + \cos(\theta))\sigma_{\text{l}} =  2 \sqrt{\sigma_{\text{s}}^{\text{lw}}\sigma_{\text{l}}^{\text{lw}}} + 2\sqrt{\sigma_{\text{s}}^{\text{ab}}\sigma_{\text{l}}^{\text{ab}}}
\end{equation}

Pro další zjednodušení nám nezbývá, než si půjčit nějaké poznatky z chemie. Teflon má například povrchovou energii pouze disperzního charakteru, tj. $\sigma_{\text{s}} = \sigma_{\text{s}}^{\text{lw}}$ a na pravé straně rovnice tak bude vystupovat jediná odmocnina. Dál můžeme jako kalibrační kapalinu použít metylenjodid (CH$_{2}$I$_{2}$), který má povrchovou energii rovněž čistě disperzního charakteru.

\begin{align}
  (1 + \cos\theta_{\text{H}_2\text{O}}) \sigma_{\text{H}_2\text{O}} &= 2 \sqrt{\sigma_{\text{s}}^{\text{lw}} \sigma_{\text{H}_2\text{O}}^{\text{lw}}} \\
  (1 + \cos\theta_{\text{kal}}) \sigma_{\text{kal}} &= 2 \sqrt{\sigma_{\text{s}}^{\text{lw}} \sigma_{\text{kal}}}
\end{align}

Po vydělení obou rovnic pro poměr disperzní a celkové povrchové energie vody platí

\begin{equation}
\frac{ \sigma^{lw}_{\text{H}_2\text{O}} }{ \sigma_{\text{H}_2\text{O}} } =
\frac{ \sigma_{\text{H}_2\text{O}} }{ \sigma_{\text{kal}} }
( \frac{1 + \cos \theta_{\text{H}_2\text{O}}}{ 1 + \cos \theta_{\text{kal}} } )
\end{equation}

\section{Výsledky měření}

\subsection{Absolutní měření viskozity}

Pomocí katetometru jsem změřil výškový rozdíl h vyústění trubičky a úrovně kapiláry 

\begin{equation}
h = 11.43 cm
\end{equation}

\subsection{Stanovení nepřímo měřené veličiny}

\section{Závěr}

\end{document}

%Povrchové napětí, které pozorujeme, je nutně důsledkem přitažlivých mezimolekulárních sil uvnitř kapalin, neboť k odhalení většího počtu molekul a ke zvětšení povrchu kapaliny je zapotřebí vykonat práci
%
%\begin{equation}
%dW = Fdx = \sigma l dx = \sigma dS.
%\end{equation}
%
%Povrchové napětí je skrz tento vztah zároveň veličina vyjadřující plošnou hustoty energie, kterou kapalina získává zvětšením povrchu.
%
%\begin{equation}
%\sigma = \frac{dW}{dS}
%\end{equation}
%
%V povrchové vrstvě ale může být mezimolekulárních interakcí mnoho a na rozhraní dvou látek se můžou působící síly lišit. Podle Fowkese a van Osse et al. tyto síly aditivně přispívají k celkovému povrchovému napětí a všechny lze zahrnout do dvou větších kategorií acidobazikých interakcí a Van der Waalsových sil
%
%\begin{equation}
%\sigma = \sigma^{\text{lw}} + \sigma^{ab}.
%\end{equation}
%
%Vzájemnou interakci mezi dvěma fázemi s, l popisuje adhezní práce, pro kterou platí Berthelotovo kombinační pravidlo
%
%\begin{equation}
%W_{\text{sl}}^{\text{a}} = 2 \sqrt{\sigma_{\text{s}}^{\text{lw}}\sigma_{\text{l}}^{\text{lw}}} + 2\sqrt{\sigma_{\text{s}}^{\text{ab}}\sigma_{\text{l}}^{\text{ab}}} 
%\end{equation}
%
%a mezifázovou povrchovou energii (tj. i např. povrchové napětí na rozhraní mezi dvěma kapalinami)
%můžeme dopočítat dosazením (3) do Duprého rovnice
%
%\begin{align}
%  \sigma_{\text{sl}} &= \sigma_{\text{s}} + \sigma_{\text{l}} - W_{\text{sl}}^{\text{a}} \\
%  \sigma_{\text{sl}} &= \sigma_{\text{s}} + \sigma_{\text{l}} - 2 \sqrt{\sigma_{\text{s}}^{\text{lw}}\sigma_{\text{l}}^{\text{lw}}} + 2\sqrt{\sigma_{\text{s}}^{\text{ab}}\sigma_{\text{l}}^{\text{ab}}}.
%\end{align}
%
%
