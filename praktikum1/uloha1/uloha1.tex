\documentclass[a4paper,11pt]{article}

% Kodovani (cestiny) v dokumentu: utf-8
%\usepackage[cp1250]{inputenc}	% Omezena stredoevropska kodova stranka, pouze MSW.
\usepackage[utf8]{inputenc}	% Doporucujeme pouzivat UTF-8 (unicode).

\usepackage[margin=2cm]{geometry}
\newtoks\jmenopraktika \newtoks\jmeno \newtoks\datum
\newtoks\obor \newtoks\skupina \newtoks\rocnik \newtoks\semestr
\newtoks\cisloulohy \newtoks\jmenoulohy
\newtoks\tlak \newtoks\teplota \newtoks\vlhkost

\jmenopraktika={Fyzikální praktikum 1}
\jmeno={Lukáš Lejdar}
\datum={27. února 2024}
\obor={F}
\skupina={Út 16:00}

\cisloulohy={1}
\jmenoulohy={Měření hustoty}

\tlak={101{,}35}
\teplota={24,1}
\vlhkost={26,6}


%%%%%%%%%%% Uzitecne balicky:
\usepackage[czech]{babel}
\usepackage{graphicx}
\usepackage{amsmath}
\usepackage{xspace}
\usepackage{url}
\usepackage{indentfirst}

%%%%%% Zamezeni parchantu:
\widowpenalty 10000 \clubpenalty 10000 \displaywidowpenalty 10000
%%%%%% Parametry pro moznost vsazeni vetsiho poctu obrazku na stranku
\setcounter{topnumber}{3}	  % max. pocet floatu nahore (specifikace t)
\setcounter{bottomnumber}{3}	  % max. pocet floatu dole (specifikace b)
\setcounter{totalnumber}{6}	  % max. pocet floatu na strance celkem
\renewcommand\topfraction{0.9}	  % max podil stranky pro floaty nahore
\renewcommand\bottomfraction{0.9} % max podil stranky pro floaty dole
\renewcommand\textfraction{0.1}	  % min podil stranky, ktery musi obsahovat text
\intextsep=8mm \textfloatsep=8mm  %\intextsep pro ulozeni [h] floatu a \textfloatsep pro [b] or [t]

% Tecky za cisly sekci:
\renewcommand{\thesection}{\arabic{section}.}
\renewcommand{\thesubsection}{\thesection\arabic{subsection}.}
% Jednopismenna mezera mezi cislem a nazvem kapitoly:
\makeatletter \def\@seccntformat#1{\csname the#1\endcsname\hspace{1ex}} \makeatother
%
\newcommand{\vsn}[4]{\ensuremath{#1 =} #2(#3)\,#4}
\newcommand{\vrn}[6]{\ensuremath{#1 =} (#2 $\pm$ #3)\,#4 ($p=$ #5\,\%, $\nu=$ #6)}


%%%%%%%%%%%%%%%%%%%%%%%%%%%%%%%%%%%%%%%%%%%%%%%%%%%%%%%%%%%%%%%%%%%%%%%%%%%%%%%
% Zacatek dokumentu
%%%%%%%%%%%%%%%%%%%%%%%%%%%%%%%%%%%%%%%%%%%%%%%%%%%%%%%%%%%%%%%%%%%%%%%%%%%%%%%

\begin{document}

\thispagestyle{empty}

{
\begin{center}
\sf 
{\Large Ústav fyziky a technologií plazmatu Přírodovědecké fakulty Masarykovy univerzity} \\
\bigskip
{\huge \bfseries FYZIKÁLNÍ PRAKTIKUM} \\
\bigskip
{\Large \the\jmenopraktika}
\end{center}

\bigskip

\sf
\noindent
\setlength{\arrayrulewidth}{1pt}
\begin{tabular*}{\textwidth}{@{\extracolsep{\fill}} l l}
\large {\bfseries Zpracoval:}  \the\jmeno & \large  {\bfseries Naměřeno:} \the\datum\\[2mm]
\large  {\bfseries Obor:} \the\obor  \hspace{40mm}  {\bfseries Skupina:} \the\skupina %
&\large {\bfseries Testováno:}\\
\\
\hline
\end{tabular*}
}

\bigskip

{
\sf
\noindent \begin{tabular}{p{4cm} p{0.6\textwidth}}
\Large  Úloha č. {\bfseries \the\cisloulohy:} \par
\smallskip
$T=\the\teplota$~$^\circ$C \par
$p=\the\tlak$~kPa \par
$\varphi=\the\vlhkost$~\%
&\Large \bfseries \the\jmenoulohy  \\[2mm]
\end{tabular}
}

\vskip1cm

\section{Úvod}

Měříme romzěry a hmotnost dutého válečku. Jeho hustotu spočítáme použitím následujícího vztahu

\begin{equation}
 \rho = \frac{m}{V} = \frac{4m}{\pi h (D^2 - d^2)} \label{eq:F}   
\end{equation}

,kde $\rho$ je hustota, $m$ hmotnost, $h$ výška válečku, $D$ průměr a $d$ vnitřní průměr válečku.

\section{Postup měření}

K měření jsme použili tyto přístroje
\begin{itemize}
  \item laboratorní váhy - k měření hmotnosti válečku
    \item posuvné měřítko (nejmenší dílek 0,02\,cm) -- k měření průměru a vnitřního průměru
    \item mikrometr (nejmenší dílek 0,01\,cm) - k měření výšky válečku
\end{itemize}

\section{Výsledky měření}

\subsection{Měření hmotnosti válečku}
Hmotnost válečku byla stanovena laboratorními váhami s citlivostí (d = 0.0001 g), 
a ověřovacím dílkem (e = 0,001 g). Nejistotu měření určíme jako $u_B = \frac{e}{3}$

\begin{equation}
m = 172.9962\ (4)\ g
\end{equation}

\subsection{Měření geometrických rozměrů}
Naměřené hodnoty geometrických rozměrů jsou spolu s aritmetickým průměrem a nejistotou uvedeny v~tabulce \ref{tb:mer}.


\begin{table}[ht]
\centering

\begin{tabular}{| r | r | r | r |}
\hline
i & průměr [mm] & vnitřní průměr (mm) & výška [mm] \\
\hline\hline
1 & 44.01 & 9.20 & 15.44 \\
2 & 44.02 & 9.22 & 15.40 \\
3 & 43.98 & 9.42 & 15.30 \\
4 & 43.96 & 9.20 & 15.36 \\
5 & 43.96 & 9.30 & 15.46 \\
6 & 43.94 & 9.28 & 15.48 \\
7 & 43.96 & 9.30 & 15.34 \\
8 & 43.92 & 9.30 & 15.42 \\
9 & 44.02 & 9.32 & 15.44 \\
10 & 43.98 & 9.42 & 15.39 \\
\hline\hline
x & 43.98 (1) & 9.30 (2) & 15.40 (2) \\\hline

\end{tabular}
\caption{Naměřené hodnoty...}
\label{tb:mer}
\end{table}

\subsection{Stanovení nepřímo měřené veličiny}
Hustotu $\rho$ jsme stanovili z přímých měření podle rovnice (\ref{eq:F}). 
Nejistota měření $\rho$ byla stanovena pomocí vztahu
\begin{eqnarray}
  u(\rho) &=& \rho \sqrt{(\frac{u(m)}{m})^2 + (\frac{u(h)}{h})^2 + \frac{(2Du(D)^2 + (2du(d))^2 }{(D^2-d^2)^2}} \\
\end{eqnarray}

Výsledek měření se standardní nejistotou

\begin{center}
\vsn{\rho}{7741}{6}{kg m$^{-3}$}.
\end{center}

Výsledek měření s rozšířenou nejistotou

\begin{center}
  \vrn{\rho}{7.74}{0.03}{ $10^3$kg m$^{-3}$ }{99,73}{9}
\end{center}
Relativní rozšířená nejistota je rovna 0.4 \%.
%$D=( \pm )\,\text{cm} \quad (p=0{,}6827, \nu= 9)$\\[1em]

%$l=2099(1)\,\text{dm}$\\

\section{Závěr}
Materiál, ze kterého je váleček vyrobený neznáme, ale vzhled i změřená hustota nejlépe odpovídají 
tabulokovým hodnotám litin železa (7580-7860) kg m$^{-3}$


\begin{thebibliography}{0}
\bibitem{navody} Bochníček a kol. \textit{Fyzikální praktikum 1, návody k ulohám.} Brno 2024.\\ Dostupné z~\url{https://monoceros.physics.muni.cz/kof/vyuka/fp1_skripta.pdf}.   
\bibitem{tabulky} Hustota pevných látek. Dostupné z~\url{http://www.converter.cz/tabulky/hustota-pevne.htmf}.   

\end{thebibliography}


% Nakonec nezapomeňte projet text programem vlna nebo vlnka, např.
% 	vlna -m -l -n mojeuloha.tex
% nebo zkontrolovat a opravit jednopísmenné předložky na koncích řádků ručně.


\end{document}
